\begin{abstract}
We develop a method to enable collaborative modelling of gravitational
lenses and lens candidates by experienced but non-professional lens
enthusiasts.  It uses an existing free-form modelling program (GLASS)
but provides the modelling input in a novel way, via a user-provided
diagram that is essentially a sketch of an arrival-time surface. 

An implementation (\spl) is tested in a modelling challenge using 29
simulated lenses out of a larger set created for a citizen-science
lens search (\sw).  We find that amateur lensers inferred the image
parities and time ordering consistently in some lenses, while tending
to make errors in other lenses, depending on the image morphology.
Errors in image parity and time ordering lead to large errors in the
mass distribution, but the enclosed mass is more robust.  Shortcomings
include (a)~lens profiles that tend to be systematically shallower
than the lenses, and consequent over-estimation of the Einstein radius
by $\sim25\%$, (c)~lack of detailed source modelling.  Ideas for
improvement are discussed.
\end{abstract}

\begin{keywords}
\end{keywords}

\section{Introduction}

The first two discoveries of gravitational lenses
\citep{1979Natur.279..381W,1980Natur.285..641W}, where a galaxy images
a background quasar into two (four), brought in their wake the first
work on lens modelling
\citep{1981ApJ...244..723Y,1981ApJ...244..736Y}.  For the second lens
(PG1115+080), mass models scored an early success with the prediction
that one of the lensed images seen would split further into two at
higher resolution.  That galaxies must sometimes cause had long been
expected \citep{1937ApJ....86..217Z}, and it had even been argued that
the phenomenon could help measure cosmological parameters
\citep{1964MNRAS.128..307R,1966MNRAS.132..101R}.  But apparently
nobody was expecting that lenses would need detailed modelling.
The first observations, however, immediately stimulated models.
The reason for that lies in the image separation. Recall that image
separations are of order the angular Einstein radius
\begin{equation}
\sim \left(\frac{4GM}{c^2 D_L}\right)^{1/2}
\simeq 0.1'' \left(\frac{M}{M_\odot}\right)^{1/2}
             \left(\frac{D_L}{\rm pc}\right)^{-1/2}
\end{equation}
where $D_L$ is the distance to the lens, and $M$ its mass.  For
stellar-mass lenses, the image separation is much larger than the size
of the lensing mass, so lenses are well approximated by point masses.
In galaxy lenses, the image separation is comparable to the size of
the galaxy; typically the lensed images are seen through the galaxy
halo.  Hence the lensed images depend on the detailed mass
distribution of the lensing galaxy.  Galaxy lenses demand modelling.

Since the early discoveries, more than 400 secure lenses are now known
(see Figure \ref{fig:masterlens}).  Modelling of the mass distribution
is part of any research using lenses, but so far no modelling study
has spanned all known lenses.  The largest single one
\citep{2009ApJ...703L..51K} models 58 lenses to infer the distribution
of dark matter around galaxies.  In other work,
\cite{2011ApJ...740...97L} combined lens models of 21 galaxies with
models of their stellar populations, to find the relation between
stars and dark matter, and \cite{2014MNRAS.437..600S} modelled 18
time-delay lenses together to infer cosmological parameters.

\begin{figure}
\centering
\includegraphics[width=\hsize]{fig/lenssky}
\caption{Sky distribution of 423 published lenses considered secure
  (from the Masterlens catalog at the University of Utah, maintained
  by Joel Brownstein and Leonidas Moustakas).  The map is in
  Hammer-Aitoff projection, with North up and $\rm RA=0$ in the
  middle. The empty swathes to the left and right of centre are the
  Milky Way. {\em Should we drop this figure?  It's not relevant to
    modelling, but as there seems to be no publicly available summary
    of all known lenses, it does help set context.}}
\label{fig:masterlens}
\end{figure}

Surveys now under way aim to increase the inventory of lenses another
ten or a hundred fold.  Among these is \sw (Marshall et al, in prep,
More et al, in prep).  \sw is a citizen science project\footnote{\tt
  http://www.spacewarps.org} in which volunteers are presented
sky-survey images and invited to identify lens candidates.  Simulated
lenses are mixed in with the data, both to help train volunteers on
what to look for, and to provide measures of the effectiveness of the
search.  The motivation for \sw is to enable volunteers, some of whom
had previously serendipitously identified lens candidates on earlier
citizen-science surveys, to make discoveries missed in automatic
searches by software robots.  Robots are very good with clean lensing
system in uncrowded fields with high signal-to-noise, but in general
test situations \citep{2009ApJ...694..924M} robots miss lenses (low
completeness) or contaminate the results with non-lenses (low purity).
\sw introduced its first tranche of survey data, in May 2013, from the
Canada-France-Hawaii Telescope Legacy Survey, or CFHTLS
\citep{Coupon2009}, covering $\simeq172$ square degrees (0.4\% of the
sky) was divided into tiles of $\simeq 80''\times80''$ ($440\times
440$ pixels).  Each such tile was shown to ten volunteers.  Despite
the rarity of detectable lenses ($<10^{-3}$ per tile), the \sw
collective soon found lens candidates, both rediscoveries already
known from robots \citep{Gavazzi2012, More2012ApJ} and possible new
lenses.

The encouraging early results prompted the question: could modelling
of the lenses also be done by volunteers?  If so, modelling could help
assess prioritise lens candidates at an early stage, which would be
very useful with new wide-field and sensitive surveys, which will
yield thousands of lens candidates. There are several software tools
for lensing modelling available, and work has been done on generic
interfaces \citep{2014A&C.....5...28L}.  Some early designs for \sw
included a prototype lens modelling tool \citep{2010AAS...21543527N}.
Moreover, some {\em Spacewarps\/} volunteers are quite experienced
from earlier projects, having spent a thousand hours or more with
data, and are very interested in more demanding projects.  The
interests of citizen-science communities are just beginning to be
studied \citep[e.g.,][]{2013AEdRv..12a0106J}, but it is clear that
some volunteers welcome open-ended challenges, and sometimes these
have led to new scientific results: one example is the discovery of an
exceptional extra-solar planet \citep{2013ApJ...768..127S}; another is
the development of new algorithms for protein folding
\citep{Khatib22112011}.  All these are grounds for optimism.  There
is, however, a basic difficulty in strong gravitational lensing.
Lensed images do not look much like their source, and still less do
they resemble the lensing-mass distribution.  To model a lens, one
needs to either do a lot of random guessing, or have a good intuition
for what works.

In this paper, we propose a way around the difficulty, and report on a
modelling test on \sw using simulated lenses.  The three following
sections are devoted to the concept, the implementation, and tests
respectively.

In \secref{Fermat} we introduce a markup system for lensed images,
which we call a ``spaghetti diagram''.  A spaghetti diagram resembles
the visible image system, in a cartoon-like way, and at the same time
it encodes the basis of a mass model.  This supplies an intuitive link
between the image system and the mass distribution, which look
frustratingly different from each other.  Spaghetti diagrams are
essentially the saddle-point contours originally introduced to
gravitational lensing by \cite{1986ApJ...310..568B} as a way of
classifying lensed images.  They are sometimes shown as part of the
output of lens models \citep[for
  example][]{2001ApJ...557..594R,2003ApJ...590...39K,Lubini2012}.  In
the present work, however, spaghetti diagrams are the {\em input\/}
through which the modeller tells the program what to do.

In \secref{SpaghettiLens} we describe the \spl program, which
implements everything.  \spl is an extension of the GLASS framework
for modelling lenses \citep{2014arXiv1401.7990C}.  We will not go into
software details in this paper, concentrating instead on lens
modelling per se, but we remark that \spl is designed to be friendly
to the forum style of citizen-science projects, and enable incremental
model refinement by different people, without sacrificing any features
of GLASS.

In \secref{mod_challenge} we describe a modelling challenge where a
diverse sample of 29 simulated lenses was modelled multiple times
using \spl.  The models are then examined in two ways.  One was
whether the spaghetti diagram was correct.  The other was the recovery
of the Einstein radius.  In addition, we show some visual comparisons
of actual and recovered lens shape and radial profile, and identify
some areas to improve, but do not compute statistics on these aspects.
Profile and shape recovery with GLASS has been studied in more detail
in \citep{2014arXiv1401.7990C}.

\secref{todo} concludes, with the general outlook and next steps.

\section{Fermat's principle and spaghetti diagrams} \label{sec:Fermat}

We now explain the lensing theory relevant to \spl, following the
formulation of gravitational lensing in terms of Fermat's principle by
\cite{1986ApJ...310..568B}.

\subsection{Geometrical and gravitational time delays} 

Consider a lens at some redshift $z_L$ and let $(x,y)$ be planar
coordinates at the lens, transverse to the line of sight.  Let
$\Sigma(x,y)$ be the mass per unit area, i.e., density projected along
the line of sight.  This surface density is often given in a
dimensionless form
\begin{equation} \label{eq:kappa}
\kappa(x,y) \propto \Sigma(x,y)
\end{equation}
called the convergence.  And let there be light, in the form of a more
distant source, at redshift $z_S$, behind point $(x_s,y_s)$ on the
lens.

We now imagine a virtual photon flying from the source to some $(x,y)$
on the lens, then changing direction and coming to the observer.  Such
a direction change would increase the light travel time compared to
coming through $(x_s,y_s)$.  The increased light travel time from the
geometry of deflection would be
\begin{equation} \label{eq:tgeom}
\tgeom(x,y) \propto (x-x_s)^2 + (y-s_y)^2 \,.
\end{equation}
assuming the delay is small compared to the total light travel time.

An additional delay of the photon comes from travelling through the
curved spacetime at the lens.  This gravitational time delay $\tgrav$
is related to the mass distribution of the lens.  The relation is
generally written as a two-dimensional Poisson equation, but an
alternative expression, avoiding calculus, is as follows.  The value
of $\tgrav$ through $(x,y)$ equals its average value on the
circumference of a small circle centred at $(x,y)$, plus a constant
times the mass within that circle.  The constant is $2G/c^3$ times the
cosmological expansion factor $(1+z_L)$. Thus
\begin{equation} \label{eq:tgrav}
\tgrav(x,y) = \left\langle \tgrav(x_\subcirc,y_\subcirc) \right\rangle
              + (1+z_L) \frac{2G}{c^3} M(x_\subbullet,y_\subbullet) \,.
\end{equation}
We have used $(x_\subcirc,y_\subcirc)$ to denote the circumference of
a circle, and $(x_\subbullet,y_\subbullet)$ to indicate the integrated
mass within the circle.

The light travel time of a virtual photon is therefore longer by
\begin{equation}  \label{eq:tarriv}
t(x,y) = t_{\rm geom} + t_{\rm grav} \,.
\end{equation}
than it would have been with no lens present.  Real photons take paths
that make $t(x,y)$ extremal, that is, having a minimum, maximum or
saddle point (Fermat's principle).

The proportionality factors in \eqref{eq:kappa} and \eqref{eq:tgeom}
depend on the redshifts and cosmological parameters, and are given in
the Appendix~\ref{more-theory}.

\begin{figure}
\centering
\includegraphics[width=\columnwidth]{fig/arriv_0}
\includegraphics[width=\columnwidth]{fig/arriv_1}
\includegraphics[width=\columnwidth]{fig/arriv_2}
\caption{Perspective views and contour maps of example arrival time
  surfaces.  Contours are coloured in rainbow order (red: least delay,
  violet: highest delay).  {\em Upper panel:\/} No lens, hence showing
  the parabolic shape of the geometrical time delay.  The image would
  be at the bottom, coinciding with the source. {\em Middle panel:\/}
  A circular lensing mass (offset from the source) has been added,
  which has pushed the minimum to one side and introduced a maximum
  and saddle point, each corresponding to an image.  The saddle-point
  is characterized by a self-crossing ``spaghetti'' contour. {\em
    Lower panel:\/} An elongated lensing mass has been added.  There
  are now two minima, two saddle points, and a maximum, each
  corresponding to an image.  Note the two self-crossing spaghetti
  contours associated with the saddle points.}
\label{fig:arriv}
\end{figure}

\subsection{Arrival-time contours} \label{sec:arriv}

The full function $t(x,y)$, also known as the arrival-time surface,
applies to virtual photons.  In other words, it is an abstract
construct and not itself observable.  But observable image positions
can be derived from the arrival-time surface, so visualising the
surface is useful.  \figref{arriv} does so.  In this figure, a
maximum, if present, is easy to see.  To locate mimima and
saddle-points, however, one needs to examine the contours of equal
arrival time.  A saddle point is characterised by a contour crossing
itself, forming an~X.  Mimima, on the other hand, have contours
looping around them, as do maxima.

The saddle-point contours which form an X are especially interesting,
because they set the overall topography of the arrival-time surface.
The obviously give the locations of the saddle points, and roughly
localise the minima and maxima as well.  If more precise locations of
the minima and maxima are added, the whole arrival-time surface is
already approximately known.  Since the arrival-time surface has an
exact relation to the lens-mass distribution and the source position,
in effect the mass distribution is also automatically specified.  In
other words, a simple sketch of saddle-point contours along with
locations of minima and maxima ---which we call a spaghetti diagram---
is implicitly already an approximation to a lens-mass distribution.

The preceding assumes a point source.  To get an idea of what an
extended source would do, let us imagine moving the original source
slightly.  The contours of constant arrival time will naturally move
slightly, and so will the images.  The movement of the contours will
be most noticeable where the contours are far apart, that is where the
arrival-time surface is nearly flat.  As is evident from
\figref{arriv}, this is the region where the minimum and saddle points
lie, or near the images.  In this region, points on the source that
are close together produce images that are comparatively far apart.
In other words, the image is highly magnified.  In summary, lower
curvature in the arrival-time surface for a point source implies
larger magnification of an extended source.  Conversely, where the
arrival-time surface is strongly curved, the image will be
demagnified.  We see from \figref{arriv} that the arrival-time surface
tends to be highly curved near the maximum.  Hence maximum tend to be
demagnified.  In practice, maxima of the arrival time are nearly
always too faint to see. The minima and saddle points dominate.

\section{A lens-modelling program} \label{sec:SpaghettiLens}

\spl is not part of \sw, but it makes use of the \sw infrastructure,
in particular, the image database and the forum.  The forum is
essential essential for initial contact with interested members of the
\sw community.  Lens modelling from \sw works, in practice, in the
usual style of medium-sized astronomical collaborations, with
videoconferencing and in-person meetings where possible, but is
reported in summary on modelling threads on the forum, and anyone
interested is welcome to join at any time.  Another aspect of \sw (and
the Zooniverse family of projects in general) that \spl adopts, is to
keep all user interaction in a web browser, with intensive computing
done on a server.  Several interesting design and implementation
issues arise from these considerations, but we will not go into them
in this paper, concentrating instead on aspects specific to lens
modelling.

Modelling with \spl involves three stages, markup of the image,
followed by intensive numerics in the background, followed by review
of diagnostics and possible discussion.  We now describe each of
these.

\subsection{Image markup}

One begins by going to the \spl page and entering the number of a \sw
image tile.  \spl presents the image, along with zoom and pan options
and a markup tool to construct a spaghetti diagram.  The human
modeller now has to make an educated guess for the topography of the
arrival-time surface, and input the locations time-ordering of the
maxima, minima, and saddle-points.  The markup tool \citep[which is
  inspired by Figure~6 of][and is like that figure made interactive
  and overlaid on data]{1986ApJ...310..568B} lets the modeller enter
the information by sketching saddle-point contours.  Examples can be
seen in the upper-left panels of Figures \ref{fig:6941} to
\ref{fig:7022}.  The loops in the markup tool were the origin of the
``spaghetti'' metaphor.

The exact placement of the loops in a spaghetti diagram has no
significance, only the connections, and the heirarchy of which loop is
inside which matters.  The loops are there simply to help modeller's
intuition.  The user does not need to think explicitly the image
parities (though the markup tool provides this information using
colour codes) or about time-ordering, or worry about the odd-image
theorem.  The markup tool allows only valid lensing configurations to
be entered.

As implemented so far, \spl assumes that the lens is dominated by a
single galaxy.  Only one maximum is permitted, and it is taken as the
centre of the main lensing galaxy.  The user can, however, mark
additional minor galaxies.  The latter are modelled as point
masses.

\subsection{Numerics}

Having sketched a spaghetti diagram, the user orders a model, and \spl
sends the input to GLASS, which runs server-side as it is
compute-intensive.  The task of GLASS is to find a mass distribution
$\kappa(x,y)$ that exactly reproduces the given locations of the
maximum, minima and saddle points. Now, this criterion by itself is
extremely under-determined --- there are infinitely many mass
distributions that will reproduce a given set of maxima, minima and
saddle points, but typically they (a)~produce lots of extra images,
and (b)~look very unlike galaxies.  Additional assumptions (a prior)
are necessary.  GLASS uses the following prior
\citep[cf.]{1997MNRAS.292..148S,2008ApJ...679...17C}.
\begin{enumerate}
\item The mass distribution is built out of non-negative tiles of
  mass.  (Sometimes these tiles are called mass pixels, but we should
  emphasize that they are unrelated to image pixels, and are much
  larger.)
\item There is a notional lens center, say $(x_0,y_0)$ which is
  identified with the maximum of the arrival time.  The source can
  have an arbitrary offset with respect to the lens center.
\item The mass distribution must be centrally concentrated, in two
  respects.  First, the circularly averaged density must fall away
  like $$ \left[(x-x_0)^2+(y-y_0)^2\right]^{-1/2}$$ or more steeply.
  Second, the direction of increasing density at any $(x,y)$ can point
  at most $45^\circ$ away from $(x_0,y_0)$.
\item The lens must be symmetrical with respect to $180^\circ$ rotations
  about $(x_0,y_0)$.  This symmetry assumption can be relaxed if the
  user wishes.
\end{enumerate}
There are still infinitely many models that satisfy both data and
prior constraints, but now they are more credible as galaxy lenses.
It is then possible to generate an ensemble of models.  The sampling
technique used by GLASS is described in \citep{Lubini2012}.
Typically, ensembles of 200 models are used.  That is to say, what we
call a \spl model is really the mean of an ensemble of 200 models, and
its estimated uncertainty is the range covered by the whole ensemble.

\subsection{Diagnostics} \label{sec:diag}

After the model ensemble has been generated, \spl post-processes it to
present results and diagnostics to the user for inspection. This takes
the form of three figures.
\begin{enumerate}
\item A synthetic image of the lensed features.
\item A contour map of the arrival-time surface $t(x,y)$.
\item A gray scale plus contour map of the mass distribution.
\end{enumerate}
The synthetic image generated by \spl assumes a simple conical source
profile.  The user can change the contrast level on the image, which
(though it is not saved) amounts to adjusting the width of the cone.
These synthetic images are still very crude and not very useful for
assessing models.  The best indicator, in practice, of whether the
modelling was successful is contour map of $t(x,y)$, with saddle-point
contours highlighted.  It is, in effect, the computer's refinement of
the spaghetti diagram input by the user.  If the arrival-time surface
looks qualitatively similar to the spaghetti diagram, that generally
indicates a successful model.  The mass distribution also provides
indications; successful models generally lead to smooth-looking mass
distribution, whereas an irregular or checkboard pattern in the mass
map signals a bad model.

After examining this feedback, the user can choose to post the model
on the \spl archive.  They can also modify their input and try again,
or discard the attempt altogether.  After archiving, there can be
discussion among modellers, through the \sw forum or by any other
means, and revision of the model.  Any archived model can be revised
by any user: they can modify the spaghetti configuration slightly or
drastically, or change options like the size of the mass tiles.
Particularly interesting lens candidates lead to trees of models in
this way.  Discussion among modellers tends to prune a model tree,
focusing attention on the most interesting models.\footnote{See
  ``Collaborative gravitational lens modelling\dots'' in {\tt
    http://letters.zooniverse.org} for an example.}

\section{A lens modelling challenge} \label{sec:mod_challenge}

We now describe a test of the lens-modelling system, under conditions
that mimic as closely as possible the modelling of real lens
discoveries.  The lenses to be modelled were the simulated lenses
(known as ``sims'') already sprinkled onto the \sw images.  Once a
small user base had grown around \spl, a modelling challenge was
announced through the \sw forum.  The challenge set consisted of 29
sims, chosen to represent the different visual morphologies of \sw
sims. Modellers then contributed 119 models for these sims (at least
two for each sim).  Models were reported on the same forum used to
model real candidate lenses.  Modellers were free to consult and
refine each other's models, but had no information on how the sims
were generated.

Once the modelling was complete, the models were compared with the
originals.  There were two main tests: a check of whether the the
spaghetti diagrams were correct for the lens in question, and a
comparison of the effective Einstein radii of the sims and the models.

\subsection{The simulated lenses} \label{sec:sims}

The \sw sims are described in detail in More et al (in prep), but
relevant here is that they were of three kinds, as follows.

\begin{enumerate}
  \item Imitating lensed quasars: having a singular elliptical
    isothermal lens (SIE) plus constant external shear, and a circular
    Gaussian source.
  \item Emulating lensed galaxies: similar to the above, but with an
    elliptical de Vaucouleurs source.
  \item Resembling cluster lenses: having a source as above, but a
    more complicated lens, with one dominant elliptical SIE and
    one or more perturbing elliptical SIEs, plus a circular NFW
    \citep{1996ApJ...462..563N,1997ApJ...490..493N} to represent
    the underlying dark matter distribution.
\end{enumerate}
The {\tt gravlens} program \citep{2001astro.ph..2340K} was used.
Formulas for the lenses appear in \cite{2001astro.ph..2341K}. The SIE
lenses follow equations (33--35) of that work, with core radius set to
zero.  The NFW lens is in equations (48) and (50), while shear is the
$\gamma$ term in equation (76).

The information in this section was not revealed to the main developer
of \spl (RK, who also chose the challenge set) or to the modellers
(EB, CC, CM, JO, PS and JW) while modelling was in progress.  That is,
the modellers had no advance knowledge of what kind of
parameterisation had been used to make the sims.  After the modelling
stage, AM released the details of the sims for post-modelling
analysis.  Results from the latter now follow.

\subsection{Some example models} \label{sec:example_models}

Of the 119 models proposed, we now discuss eight examples in some
detail.  Results from these are shown in Figures
\ref{fig:6941}--\ref{fig:7022}.  The first four of these show the most
common image morphologies, the other four explain some problem cases.

Each of Figures \ref{fig:6941}--\ref{fig:7022} figures has the
following layout.
$$ \begin{matrix}
\hbox{marked-up image} \qquad &\hbox{model synthetic image} \\
t(x,y)                        &\hbox{model } t(x,y) \\
\kappa(x,y)                   &\hbox{model } \kappa(x,y)
\end{matrix} $$
The marked-up image is a zoom of the lensed image on Spacewarps,
marked up with one or more spaghetti contours; this is the modeller's
input to \spl.  The three panels on the right show the graphical
output returned to the modeller by \spl, listed as (i), (ii), (iii) in
\secref{SpaghettiLens}.  These derive from the mean of an ensemble of
200 models generated by \spl. The model synthetic image is too crude
to be useful, because there is no fitting for the source light
distribution, as already noted in \secref{diag}.  The quality of the
synthetic images can be improved by interpolation, but as this feature
was not available during the modelling challenge, we do not show the
improved versions here.

\FloatBarrier

\begin{figure}
  \centering
  \includegraphics[width=\myplotswidth]{fig/006941_input}
  \includegraphics[width=\myplotswidth]{fig/006941_arr_time} \\
  \includegraphics[width=\myplotswidth]{fig/ASW000102p_006941_arriv} 
  \includegraphics[width=\myplotswidth]{fig/006941_spaghetti} \\
  \includegraphics[width=\myplotswidth]{fig/ASW000102p_006941_kappa}
  \includegraphics[width=\myplotswidth]{fig/006941_mass}
  \caption[result 6941 (ASW000102p)]{A simulated lens that mimics a
    lensed quasar, and model results.  The left panels derive from the
    simulation, and the right panels are \spl output.  Details of
    individual panels are in \secref{example_models}.}

  \label{fig:6941}
\end{figure}

\begin{figure}
  \centering

  \includegraphics[width=\myplotswidth]{fig/006915_input}
  \includegraphics[width=\myplotswidth]{fig/006915_arr_time} \\
  \includegraphics[width=\myplotswidth]{fig/ASW0001hpf_006915_arriv}
  \includegraphics[width=\myplotswidth]{fig/006915_spaghetti} \\
  \includegraphics[width=\myplotswidth]{fig/ASW0001hpf_006915_kappa}
  \includegraphics[width=\myplotswidth]{fig/006915_mass}

  \caption[result 6915 (ASW0001hpf)]{A four-image configuration
    typical of lensed quasars. (See \secref{example_models} for
    details.)}
  \label{fig:6915}
\end{figure}


\begin{figure}
  \centering
  
  \includegraphics[width=\myplotswidth]{fig/006990_input}
  \includegraphics[width=\myplotswidth]{fig/006990_arr_time} \\
  \includegraphics[width=\myplotswidth]{fig/ASW0004oux_006990_arriv}
  \includegraphics[width=\myplotswidth]{fig/006990_spaghetti} \\
  \includegraphics[width=\myplotswidth]{fig/ASW0004oux_006990_kappa}
  \includegraphics[width=\myplotswidth]{fig/006990_mass}

  \caption[result 6990 (ASW0004oux)]{Results from a system with an arc
    plus a counter-image, typical of lensed galaxies. (See Section
    \ref{sec:example_models} for details.)}
  \label{fig:6990}
\end{figure}



\begin{figure}
  \centering

  \includegraphics[width=\myplotswidth]{fig/006919_input}
  \includegraphics[width=\myplotswidth]{fig/006919_arr_time} \\
  \includegraphics[width=\myplotswidth]{fig/ASW0002z6f_006919_arriv}
  \includegraphics[width=\myplotswidth]{fig/006919_spaghetti} \\
  \includegraphics[width=\myplotswidth]{fig/ASW0002z6f_006919_kappa}
  \includegraphics[width=\myplotswidth]{fig/006919_mass}

  \caption[result 6919 (ASW0002z6f)]{Another configuration of arc plus
    counter-image; here the arc is closer to the lensing galaxy than
    the counter-image. (See \secref{example_models} for details.)}
  \label{fig:6919}
\end{figure}



\begin{figure}
  \centering
  \includegraphics[width=\myplotswidth]{fig/006975_input}
  \includegraphics[width=\myplotswidth]{fig/006975_arr_time} \\
  \includegraphics[width=\myplotswidth]{fig/ASW000195x_006975_arriv.png}
  \includegraphics[width=\myplotswidth]{fig/006975_spaghetti} \\
  \includegraphics[width=\myplotswidth]{fig/ASW000195x_006975_kappa.png}
  \includegraphics[width=\myplotswidth]{fig/006975_mass}
  \caption[result 6975 (ASW000195x)]{A lens with unrecovered mass
    substructure. (See \secref{example_models} for details.)}
  \label{fig:6975}
\end{figure}
  
\begin{figure}
  \centering

  \includegraphics[width=\myplotswidth]{fig/006937_input}
  \includegraphics[width=\myplotswidth]{fig/006937_arr_time} \\
  \includegraphics[width=\myplotswidth]{fig/ASW0000vqg_006937_arriv}
  \includegraphics[width=\myplotswidth]{fig/006937_spaghetti} \\
  \includegraphics[width=\myplotswidth]{fig/ASW0000vqg_006937_kappa}
  \includegraphics[width=\myplotswidth]{fig/006937_mass}

  \caption[result 6937 (ASW0000vqg)]{A sim with unrecovered
    substructure, resulting in a poor mass model. (See Section
    \ref{sec:example_models} for details.)}
  \label{fig:6937}
\end{figure}

\begin{figure}
  \centering

  \includegraphics[width=\myplotswidth]{fig/007025_input}
  \includegraphics[width=\myplotswidth]{fig/007025_arr_time} \\
  \includegraphics[width=\myplotswidth]{fig/ASW0000h2m_007025_arriv}
  \includegraphics[width=\myplotswidth]{fig/007025_spaghetti} \\
  \includegraphics[width=\myplotswidth]{fig/ASW0000h2m_007025_kappa}
  \includegraphics[width=\myplotswidth]{fig/007025_mass}
  \caption[result 7025 (ASW0000h2m)]{A four-image system with image
    parities incorrectly identified.  The model is poor, but the
    estimated Einstein radius is not bad. (See Section
    \ref{sec:example_models} for details.)}
  \label{fig:7025}
\end{figure}

\begin{figure}
  \centering


  \includegraphics[width=\myplotswidth]{fig/007022_input}
  \includegraphics[width=\myplotswidth]{fig/007022_arr_time} \\
  \includegraphics[width=\myplotswidth]{fig/ASW0000h2m_007022_arriv}
  \includegraphics[width=\myplotswidth]{fig/007022_spaghetti} \\
  \includegraphics[width=\myplotswidth]{fig/ASW0000h2m_007022_kappa}
  \includegraphics[width=\myplotswidth]{fig/007022_mass}

  \caption[result 7022 (ASW0000h2m)]{The same system as in Figure
    \ref{fig:7025}, this time with image parities correctly
    identified. (See \secref{example_models} for details.)}
  \label{fig:7022}
\end{figure}

\FloatBarrier

In these figures, the scales and alignments in the six panels are not
completely consistent: the three panels on the right exactly the same
coordinates; the middle-left and lower-left panels are consistent with
each other, but only approximately aligned with the right column,
because the latter are affected by errors in mouse placements; the
upper-left panel is zoomed out slightly, to accommodate the full
markup.  For qualitative features, which is all that
Figures~\ref{fig:6941}--\ref{fig:7022} try to convey, the slightly
varying scales and alignments are harmless.

Let us now consider these cases in turn.

\begin{itemize}

\item Figure \ref{fig:6941} shows the simplest case, with two clear
  images produced by a nearly-circular lens.  The centre of the
  lensing galaxy is a maximum, the image nearer to the galaxy is a
  saddle point, and image further away is a minimum, and these were
  correctly identified.

\item Figure \ref{fig:6990} shows an example of an arc that has split
  into three images.  This kind of configuration, with a counter-image
  close to the lensing galaxy and a more distant arc/triplet on the
  other side, generically arises from an elongated mass distribution
  when the source is displaced along the elongated direction.

\item Figure \ref{fig:6919} shows another example of an arc plus
  counter-image, but (in contrast to Figure \ref{fig:6990}) the arc is
  closer to the lens than the counter-image. This configuration arises
  if the source displacement is perpendicular to the long axis of the
  lensing mass.  Comparing the two panels in the middle row, we see
  that the modeller interpreted arc as consisting of three images,
  whereas the sim shows a single saddle point associated with the arc.
  But the identification is not really erroneous --- we just need to
  take into account that the source is extended.  In fact, in the sim
  the brightest part of the source is only doubly imaged, but the
  source extends into a region that produces four images.  In the
  $t(x,y)$ of the sim, the hairpin-bending contours are typical of
  double on the verge of splitting into a quad.

\item Figure \ref{fig:6915} shows another quad.  This kind of
  configuration arises when the mass is elongated and the source is
  displaced at an angle to the elongation.  The minima and saddle
  points are correctly identified, and the orientation of the
  ellipticity of the mass distribution is correctly reproduced.

\item Figure \ref{fig:6975} shows a lens with substructure in the form
  of a smaller secondary galaxy.  The galaxies in such group or
  cluster sims were based on galaxies visible in the images, but the
  modelers were not told in advance whether this was the case.  The
  minimum and saddle point are correctly identified.  The mass
  distribution misses the substructure, but overall appears
  reasonable.

\item Figure \ref{fig:7025} shows a quad.  In this one, the
  identification of the minima and saddle points was incorrect, and
  mass distribution comes out elongated East-West instead of
  North-South.  The mass distribution also appears somewhat jagged and
  the saddle-point contours are not as clean as in the previous
  examples; these are often indicators of a problem with the model.
  The enclosed mass is, however, none the worse --- the reason is
  probably that in a relatively symmetrical image configuration, the
  Einstein radius is quite well constrained by the images in a fairly
  model-independent way.

\item Figure \ref{fig:7022} shows another model of the same system.

\end{itemize}


