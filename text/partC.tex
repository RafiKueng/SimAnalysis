\appendix

\section{Relation to standard lensing formalism} \label{sec:more-theory}

The presentation of the arrival-time surface in \secref{Fermat}
omitted details, for the sake of a more intuitive explanation.
Specifically, the geometric time delay $t_{\rm geom}$ and the
convergence $\kappa$ were left as proportionalities (equations
\ref{eq:tgeom} and \ref{eq:kappa}), and the gravitational time delay
$t_{\rm grav}$ was given in an implicit form (equation
\ref{eq:tgrav}).  We now fill in the details.

The original formulation of the arrival-time surface appears in
equations (2.1) to (2.6) of \cite{1986ApJ...310..568B}.  Their
equations can be rearranged as follows.
\begin{equation} \label{cfBN86}
\begin{aligned}
t_{\rm geom} &= \frac{(1+z_L)}{2c} \frac{d_S}{d_L d_{LS}}
\left[ (x-s_x)^2 + (y-s_y)^2 \right] \\
\nabla^2 t_{\rm grav} &= -(1+z_L)\frac{8\pi G}{c^3} \, \Sigma(x,y) \\
\kappa(x,y) &= \frac{4\pi G}{c^2} \frac{d_L d_{LS}}{d_S}
               \times \Sigma(x,y)
\end{aligned}
\end{equation}
We symbols $d_L,d_S$ and $d_{LS}$ are angular-diameter distances,
respectively from observer to lens, observer to source, and lens to
source.  In the concordance cosmology, we have
\begin{equation}
d_{LS} = \frac{c}{H_0}\,\frac1{1+z_S} \,
\int_{z_L}^{z_S} \!\! \frac{dz}{\sqrt{\Omega_m(1+z)^3 + \Omega_\Lambda}}
%\raise.5ex\hbox{.}
\end{equation}
and similarly $d_S$ and $d_{LS}$.  We have replaced angular positions
on the sky with positions on the lens plane as
\begin{equation}
(x,y) = d_L (\theta_x,\theta_y) \,.
\end{equation}

The first line of equation \eqref{cfBN86} is $t_{\rm geom}$ from
equation \eqref{eq:tgeom} with the proportionality filled in, and with
the source offset at $(s_x,s_y)$ rather than at the origin.

The last line of equation \eqref{cfBN86} fills in the proportionality
factor in equation \eqref{eq:kappa} for the convergence (or
dimensionless surface density) $\kappa$.

The middle line of equation \eqref{cfBN86}, though very different in
form to the expression \eqref{eq:tgrav} for the gravitational time
delay, is in fact equivalent.  This can be shown by considering a
multipole expansions for the solution of the latter equation.  Another
way to see the equivalence is to write a discrete version of
\eqref{eq:tgrav}.  Introducing a small distance $\Delta$, we write
\begin{equation} \label{eq:relax}
\begin{aligned}
\tgrav(x,y) &=
\frac14 \Big[ \tgrav(x+\Delta,y) + \tgrav(x-\Delta,y) + {}\\
&\kern24pt    \tgrav(x,y+\Delta) + \tgrav(x,y-\Delta) \Big] \\
&+ (1+z_L)\frac{2G}{c^3} \, \pi\Delta^2 \, \Sigma(x,y) \,.
\end{aligned}
\end{equation}
The circular average
$\left\langle \tgrav(x_\subcirc,y_\subcirc) \right\rangle$ has
been replaced by the average over four neighbouring points, and the
enclosed mass $M(x_\subbullet,y_\subbullet)$ has been replaced by
$\pi\Delta^2\,\Sigma(x,y)$.  Now let us read equation \eqref{eq:relax}
as an iterative formula for $t_{\rm grav}(x,y)$.  It is then
recognisable as a formula for solving the two-dimensional Poisson
equation from \eqref{cfBN86} by relaxation.

\newpage

