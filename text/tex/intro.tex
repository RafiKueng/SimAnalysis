\section{Introduction}
Gravitational lenses (GLs), predicted by General Relativity (GR), can be used as a tool by astronomers to examine many properties of the cosmos.
They allow for example to estimate the masses and their distribution of galaxies, automatically including the hard to grasp dark matter.
Further more, they allow an estimate on cosmological parameters like the Hubble constant \citep{Saha2006} and the mass density \needcite.

%To achieve this, one needs to find / identify lenses and in a second step model those lenses to get involved parameters.

How to find gravitational lenses? Huge amount of image data from surveys that needs to be processed.
Even more coming in the future with new surveys \needcite
Robotic processing has been suggested \needcite, and tested \needcite, but has failed so far to be convincing \needcite.
Another suggestion involves humans, volunteers \footnote{A volunteer is considered everyone that has no background in astrophysics.}.
\sw uses this approach with great success so far.\needcite


Next step involves modeling. that needs advanced knowledge and takes a lot of time.
Too much time to be done by astronomers themselves as the results from new surveys and identifiers like \sw come in.
So there is a demand of a means to model a great number of identified lens data, that scales with increasing volume of data.

Algorithm discovery in Foldit\citep{Khatib22112011}.

The purpose of this study was to provide a means to model a large number of gravitational lenses by showing that gravitational lens modeling can be learned / done by volunteers.
We suggest, that volunteers will be as successful as professionals with modeling if provided with an easy to use tool with visual feedback (What you see is what you get, WYSIWYG) and a minimal set of instructions.
Volunteers will then crowd work (using buzz word here ;) ) / work collaborative on modeling lenses from several sources / groups at a central place.
Since this is an iterative learning process, the more involved volunteers will quickly gain knowledge that can be passed down to new volunteers.
That creates a social structure that scales well with the number of volunteers, as other projects\needcite have already shown.
Finding people working as volunteers has been shown to be successful last but not least by \sw and the whole galaxy zoo project.
To test the people's abilities, we investigated the performance of a few volunteers modeling a set of simulated lenses.
We tested the ability to correctly identify lensed images and reproduce similar mass map of the lens.
