
\section{Results} \label{sec:results}

\subsection{First test} \label{sec:results.1}

The evaluation of the volunteers performance was done manually / by hand/eye, comparing their input to the actual contour lines.
The actual contour lines were generated from the original simulation parameters.

For each model a set of binary tests was done:

\begin{enumerate}
  \item (approxPlace) have the images approximately been identified correctly (approx within of 10\% of img width)
%  \item (rightPlace) are the images placed precisely (approx. within 5\% of image width)
  \item (rightOrder) is the ordering of the images correct (sad / min / max detection and ordering)
\end{enumerate}

Additionally, ten categories of error that could happend were recorded, where each model could contain multiple of those:

\begin{enumerate}
  \item inaccurate placement in an extended arc
  \item wrongly identified sad and min in 3 image configuration
  \item identified only 3 instead of 5 images
  \item tried to model an arc with a min instead of min-sad-min
  \item PI-err (rotation by 180 degrees; in 5 image configuration, exchanged the ordering of the two saddle points)
  \item PI/2-err (rotation by 90 degree; sad$\longmapsto$min$\longmapsto$sad$\longmapsto$min$\longmapsto$sad)
  \item missed faint image(s)
  \item tried to model an arc with min-sad-min instead of only min
  \item did identify two close by images as one
  \item used 7 or more image to model a 5 image system.
\end{enumerate}

\input{tab/auto/_stats}

By looking at the numbers in table \ref{tab:stats}, we conclude that the volunteers are performing very well identifying images and putting them approximately at the right spot (test 1), with a performance of 92%.
Most of the problems where due to unclear arc-like structures. (Error 1, p=0.18)

\todo{add more detail?} Add more details like total amount of images detected / tot am images (rightPlace fraction)??

The ordering and assignment of minima / maxima / saddlepoint posed a more difficult task.
In p=59\% (N=70) of the cases the volunteers succeeded to identify the right configuration.
Most of the failures are due to error type 6 (PI/2), with N=38, p=32\%, followed by type 5 (PI) with N=7, p=6\%.
For an example of type 6 error, see \figref{6971}.










\subsection{Second test} \label{sec:results.2}



\begin{figure}[htbp]
  \centering
    \includegraphics[width=0.80\textwidth]{fig/sims/eR_4.png}
  \caption{relative Einstein Radius for each model, per sim. }
  \label{fig:eR_all}
\end{figure}




\hr

Figure \ref{fig:6941} shows a simple example.

Figure \ref{fig:6975} shows an example where substructure introduces a
complication, but model ok.

Figure \ref{fig:6937} shows a case where substructure leads to a
poorer model.

Figure \ref{fig:6975} shows a nice symmetric quad.

Figure \ref{fig:6990} shows a long-axis quad.

Figure \ref{fig:6919} shows a short-axis quad.

Figure \ref{fig:6915} shows an inclined quad.

Figure \ref{fig:6971} shows incorrect identification, but the enclosed
mass still quite well recovered.




\clearpage
