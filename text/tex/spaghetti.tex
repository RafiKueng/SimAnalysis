\section{\spl} \label{sec:SpaghettiLens}

Give a guess for the maximum, minimum and saddle points.  Program
tries find $\kappa(x,y)$ that reproduces these properties exactly, and
looks reasonably like a galaxy.  Solution not unique, an ensemble
generated.

\hr

SpaghettiLens is build on top of GLASS \citep{Lubini2012}, that builds and improves upon PixeLens \citep{Saha2004}.
It provides a web based graphical user interface that allows to trace contours with the mouse and identify extremal points by clicking on the image.
\spl then tries to find a mass distribution $\kappa(x,y)$ that reproduces the input parameters.
It provides direct visual feedback by rendering a synthetic image side by side the original image, such that users can directly compare the predictions of their model to the survey image.
It additinally provides plots of the modelled mass distribution and the contour lines of arrival time surface as feedback. 

Using this direct visual feedback, even unexperienced users can successfully model lenses using a iterative approach if they know what to look for.

The modelling process consitis of three basic steps:

\begin{enumerate}
  \item identify lensed images and separate them from other background light sources
  \item classify and order images accoring to arrival time. (local minima, maxima, saddlepoints)
  \item fine tune the arrangement, identify addidional external point masses influencing the result
\end{enumerate}

\spl assists volunteers with several features in this process.
Step 1 by supplying several images from several bands (not yet implemented in \sw).
Step 2 by restricting the user, only valid configuration can be entered, the odd number theorem\needcite is taken care of.
Step 3 by providing the visual feedback. Users can check the generated mass distribution and synthetic image.

Explanation:
Synthetic image: A rendering of the derivative of the arrival time for each pixel. This leads to a black and white image that reasonably looks like the visual appearace of the model.

Volunteers enter an assumed lens configuration and get a rendered synthetic image along with a reconstructed / simulated mass map and reconstructed / simulated contour lines.
This allows volunteers to directly see the effect of changing the parameters of the model and to iteratively approach a good solution.

In the next step, several volunteers can work together / exchange their ideas and models and try to improve previous modelling attempts by others.
This leads to a set of models organised in a branching, tree like structre for each model.
Volunteers then organise them selves to narrow down the different branches and come up with a few consensus models in the end.

\spl provides plug ins to support any data source. At the moment, a data link to \sw and to the MasterLens database\needcite are implemented.






\section{The simulated lenses} \label{sec:sims}

Three kinds of sims.

\begin{itemize}
\item {\em Quasar:\/} A singular elliptical isothermal plus constant
  external shear, with a circular Gaussian source.
\item {\em Galaxy:\/} Lens as with quasar, but elliptical de
  Vaucouleurs source.
\item {\em Galaxy:\/} Lens is circular NFW plus one dominant
  elliptical SIE and perturbating elliptical SIEs, source as with
  galaxy.
\end{itemize}
Lenses follow \cite{2001astro.ph..2341K,2001astro.ph..2340K}.
Parameters in online supplement.