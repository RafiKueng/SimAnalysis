\begin{figure}
\includegraphics[height=.3\vsize]{fig/arriv1.png}
\includegraphics[height=.3\vsize]{fig/arriv2.png}
\includegraphics[height=.3\vsize]{fig/arriv3.png}
\caption{Arrival-time surfaces, with contour levels of equal arrival
  time.  The upper surface is with no lens.  The middle surface is
  when a circular lensing mass (offset from the source) is added; a
  maximum, a minimum and a saddle point can be seen.  The last surface
  is the result of an elongated lensing mass; a maximum, two minima
  and two saddle points can be seen. \label{fig:arriv}}
\end{figure}

\section{Fermat's Principle} \label{sec:Fermat}

There are several ways to understand the formation of arcs and
multiple images in gravitational lensing.  We will follow some ideas
originally introduced by \cite{1986ApJ...310..568B}, based on Fermat's
principle.  They key to this approach is an abstract construct called
the {\em arrival-time surface.}  This surface cannot itself be
observed, but several observable quantities can be derived from it.

Consider a gravitational lens.  As in most astrophysical lensing, this
lens is `thin' along the line of sight, and effectively lies on a
plane.  Let $x,y$ be coordinates on this plane.  These coordinates
measure ordinary physical lengths (metres, etc) on the lens plane.
Let there be a point source of light at $x=0,y=0$ but far behind the
lens plane.  Now imagine a light ray coming from the source to the
lens plane $(x,y)$ and then changing its path to reach the observer.
The observer would see the source apparently behind $(x,y)$ on the
lens.  If this happened, the light ray would have taken a longer path
than having come directly to the observer.  The geometrical path
difference would be $\propto x^2 + y^2$ for small angles.  Let us
define
\begin{equation} \label{eq:Ageom}
A_{\rm geom}(x,y) = \half(x^2 + y^2) \,.
\end{equation}
As written, this looks is an area.  But we can also think of it as a
time delay introduced by the light ray bending --- with an unwritten
constant factor.  That factor is basically the lens distance times the
speed of light; the precise expression has to take the expansion of
the Universe into account, and is given in the Appendix.  The top part
of Figure~\ref{fig:arriv} shows what $A_{\rm geom}$ looks like: it is
just a paraboloid.  The minimum of the surface is at the source, and
that is where, according to Fermat's principle, the image will be.
The rest of the arrival-time surface is just an imaginary surface.

The geometrical time delay \eqref{eq:Ageom} is not the only one.  The
warping of space by a gravitational field introduces a further time
delay $A_{\rm grav}$.  The arrival time of the light ray, compared to
what it would have been with no lensing, is
\begin{equation}
\hbox{arrival time} = A_{\rm geom} + A_{\rm grav} \,.
\end{equation}
We can also express $A_{\rm grav}$ as an area, with the same implicit
constant factor that makes it a time.  The expression, however, is
more complicated.  We first introduce the notation $\nabla^2 f(x,y)$
to denote
\begin{equation}
 \frac{ f(x+\Delta x, y) + f(x-\Delta x, y) +
        f(x, y+\Delta y) + f(x, y-\Delta y) - 4 f(x,y) }
      {\Delta x \; \Delta y}
\end{equation}
in the limit of $\Delta x,\Delta y$ small.  Then we write the
projected mass density of the lens, which would be in $\rm
kg\,m^{-2}$, in special units (given in the Appendix) such that the
surface density becomes a dimensional field $\kappa(x,y)$.  Then
\begin{equation} \label{eq:poisson}
\nabla^2 A_{\rm grav}(x,y) = -2\kappa(x,y) \,.
\end{equation}
Note that $A_{\rm grav}(x,y)$ is not given explicitly in terms of
$\kappa(x,y)$, but as a differential equation that must be solved.
Equations of the type \eqref{eq:poisson} are, however, well known (as
Poisson equations in two dimensions), and there are techniques for
solving them.

What is the effect of mass on the arrival-time surface?  If the mass
distribution $\kappa(x,y)$ is circular, $A_{\rm grav}(x,y)$ will have
a peak at the centre of that circle.  But if the mass is not precisely
in front of the mass, the arrival-time surface will not be circular
any more.  The result is illustrated in the middle part of
Figure~\ref{fig:arriv}.  In addition to the maximum, there is now a
minimum and a saddle point.\footnote{A saddle point is recognizable in
  a contour map because contours do not loop around it, they tend to
  approach and then pull away.  There is one contour ---the
  saddle-point contour--- that makes an X on the saddle point itself.
  A saddle point contour may not appear on a plot (and
  Figure~\ref{fig:arriv} does not show one.  But we can tell from the
  other contours where the saddle-point contour would be.}

Here $\kappa$ is a dimensionless quantity with two distinct physical
interpretations.  On the one hand, it is the sky projected density ---
in special units, to make it dimensionless.  On the other hand,
$\kappa$ is the {\em convergence\/} of a bundle of rays.
